\documentclass[ms,a4paper]{memoir}
\chapterstyle{dash} % try also reparticle
\usepackage{ulem}   % underline
\usepackage{xcolor}
%% Polski
\usepackage[T1]{fontenc}
\usepackage[polish]{babel}
\usepackage[utf8]{inputenc}
\usepackage{marginnote}
%
%\usepackage{hyperref}
% All font size must be normal size
\renewcommand{\large}{\normalsize}
\renewcommand{\Large}{\normalsize}
\newcommand{\red}[1]{\textcolor{red!50!black}{#1}}
\newcommand{\RED}[1]{\textcolor{red!50!black}{\MakeUppercase{#1}}}
% font hyphenation
\usepackage{everysel}
\EverySelectfont{%
\fontdimen2\font=0.6em % interword space
\fontdimen3\font=0.2em % interword stretch
\fontdimen4\font=0.1em % interword shrink
\fontdimen7\font=0.9em % extra space
\hyphenchar\font=`\-% to allow hyphenation
}


%\usepackage[spacing=true,factor=1200, stretch=10, shrink=15]{microtype}

\renewcommand{\baselinestretch}{1}


\newbox\flinebox{}
\newbox\slinebox{}
\newbox\mlinebox{}
\def\duplines{\setlength\parindent{0pt}
  \setbox\flinebox\lastbox{}
  \ifvoid\flinebox\relax
  \else
  \setbox\slinebox\hbox{\copy\flinebox}
  \setbox\mlinebox\hbox{\copy\flinebox}
  \unskip\unpenalty{}
  {\duplines}

{\color{black!30} \box\flinebox\vspace*{-2.85ex}}
{\color{black!50} \makebox[\textwidth]{\hspace*{-0.25pt}\box\mlinebox}\vspace*{-2.75ex}}
{\color{black!90}  \makebox[\textwidth]{\hspace*{0.25pt}\box\slinebox}}\fi

}

\newcommand\BlurText[1]{%
  \vbox{#1\par\duplines}}


\begin{document}
\chapter{\RED{Skrót}}
\section{}
\paragraph{}
Pakaczuki jest przygodą pod realia Zewu Cthulhu, rozgrywać się powinna w międzywojniu (1920 \- 1930)
w centralnych Stanach Zjednoczonych. Przewiduje się ją na \emph{1} do 2 sesji, przy ekipie od \emph{1} do 4 graczy.\\
\section{Skrót}
Gracze na zaproszenie wuja jadą do małej miejscowości w górach (Pakaczuki).
Droga do tego miejsca jest utrudniona z powodu na rozmokłe drogi oraz pozawalane mosty.
Na miejsciu okazuje się, że wiosenne roztopy co roku zawalają jedyny most z miejscowości, a mieszkańcy składają przybyłych w ofierze.
Jedyne możliwe wyjście z terenu osady jest przez kopalnie i nieeukiesowy labirynt.
\paragraph{}
Przygoda jest aktualnie w stanie \red{alpha}.
\newpage
\tableofcontents
\newpage

\chapter{\RED{Przed rozgrywką}}

\section{Wymagania co to graczy}
Przygoda nie ma jakiś szczególnych wymagań co do graczy.
Wszelkie niezbędne przedmioty będą do znalezienia na miejscu,
a żadna z umiejętności nie jest niezbędna do pchnięcia przygody
na przód. Jedynym ważnym elementem jest fakt, że jeden z graczy
ma postać nazywającą się \emph{Evert Hampton}. Wymóg ten jest
podyktowanym treścią testamentu.
\section{Potrzebne materiały}
Mistrz gry powinien wyposażyć się w:
\begin{itemize}
\item Małą kopertę
\item Dużą kopertę (w której zmieści się mała)
\item Wydruk testamentu ze strony %\pageref{hand:testament}
\end{itemize}

\chapter{\RED{Prolog}}
\section{Tło fabularne}
Jeden z graczy (Evert Hampton) otrzymał \emph{tydzień temu} list od swojego wuja, Johna.
Wuj zaprasza go do Pakaczuki, swojego domu, na obchody \emph{Święta}.
Jest to pierwszy list od roku i nie ma w nim wzmianki o przepisie na pierogi, o który Evert prosił.
Nie jest to jendak nic dziwnego, wuj w przeszłości potrafił przestać pisać na parę miesięcy i po tym czasie odpisać jak by nic się nie stało.
Podeskcytowany możliwością spotkania krewnego, którego nie widział od ponad
siedmiu lat, Evert chętnie zabiera się do drogi.
Wuj w liście zaprasza nie tylko jego, ale jeśli ma jeszcze z dwóch trzech przyjaciół, to spokojnie się zmieścicie.


\section{Otwarcie}
Postaci podróżują już jeden lub dwa dni do celu.
Podróż przebiega w miare normalnie. Ford T którym się przemieszczają nie zawodzi, a pogoda
dopisuje. Mimo że jest \emph{przełom marca i lutego} słońce dogrzewa i jedzie się stosunkowo
przyjemnie. Właśnie dojechali do zajazdu w małej
miejscowości pod górami. [TODO] bo tak nazywa się ta miejscowość. Jest praktycznie ostanim miejscem z ogolnie rozumianą cywilizacją.
Miejscowość ta ma poza zajazdem kilka domów, sklep i warsztat/stację benzynową.
Zwróć uwagę graczy, że wedle mapy stąd do Wuja jest jeszcze 2h.
Jak wezmą poprawkę na pogodę i średnie warunki drogowe, maksymalnie 4h jazdy.
Trzeba zrobić wrażenie, że nie da się już skoczyć tam teraz po nocy, ale rano to moment i są na miejscu.

\subsection{Miejsce: Zajazd}
Zajazd jest wielkości dużego domu.
Sprawia wrażenie miejsca raczej przytulnego.
Ma dwa piętra i parter.
Na piętrach jest 8 pokoi do spania.
A parter przeznaczony jest na jadalnie, recepcję i pokój oczekiwania.
Z uwagi na fakt, że gracze przyjeżdzają totlanie poza jakimkolwieks sezonem, są jedymi goścmi.
Miejsce to prowadzi rodzina mieszkająca parę domów dalej, akurat jak gracze przyjadą na recepcji
jest Pani Hanna Sevani.
\marginnote{|Osoba: Hanna Sevani}
Hanna jest starszą ale jeszcze aktywną kobietą, pod 50 lat.
Jest aż zbyt słodka, może odnieść się do postaci per `kochanienki' jeśli jej pozwolimy.
Pyta co gracze chcą na śniadanie, ale z uwagi na prywatne powody miękko upiera się by to było o 7 rano nie później.
\\
Tu mamy scene śniadania w zajeździe.
Jest to dobre miejsce na opisanie wyglądu postaci i krótką pogawędkę przy jajecznicy.

Przy samym wyjeździe zwróć uwagę, że piękny wschód słońca przeszedł w równie piękny poranek.
Ptaszki śpiewają a rzadkie przebiśniegi i pęki drzew wydają się im zalewać cały świat zielonoscią.
Pilnuj by nie zabrali więcej niż jednej kanapki na drogę, może wspomnij że Wujo pewnie ich ugości.

Wyjechali?
\chapter{\RED{Droga do Pakaczuki}}

Jak tylko zajazd zniknie im za tylnią szybą na tyle by nie myśleli wracać, zasnuj niebo szarymi chmurami.
Jest bardzo ciężka pogoda, wieje, mży, a koła Forda wprawionego w takich warunkach uślizgują się na błocie pośniegowym.
Turlają się tak z 6 do 8 godzin.
Opisz jak bardzo jest im zimno.
Jak bardzo czują że ich kości zamarzają.
Jak przeklinają tą przemoczoną mapę na której już nic nie widać.
Jeśli zarzucą, że mają takie umiejętności jak ``Nawigacja'' wyjaśnij im, że nie robiłeś testów bo nawigacja idzie im świetnie.
Tylko po drodze mijają już któryś zawalony most i próbują na około.
Nie zatrzymali się też po drodze, ugotować coś albo zrobić miejsce do spania bo za każdym razem wierzyli, że to już za tym wzgórzem bo mapa pokazuje.
Jak już osłabiłeś ich ducha.
Jak już faktycznie czują beznadzieje sytułacji czas wprowadzić kolejnego BNa.
\marginnote{|Osoba:Wolny Jeneń}
Z kierunku w którym jadą nadjeżdza na koniu indianin.
Jest pozytywnie nastawiony i macha do graczy.
Odziany jest w Jeansy i Ponczo.
Gdy podjedzie zauważa, że pewnie się zgubili i dalsza jazda tą drogą nie ma najmniejszego sensu, tamten most też jest zawalony.
Po usłyszeniu, że jadą do Pakaczuki, tłumaczy że zna drogę, ale teraz to by się tam pieszo by nie pakował, a automobilem to już na pewno.
Zauważa też, że idzie burza, proponuje w geście dobrej woli, że ich przenocuje i nalega by iść za nim.
Może poprosić o podwiezienie ``futer'', które udało mu się upolować na bagażniku samochodu.
Spokojnie ma koc którym wyłożył grzbiet konia więc nie ubrudzi.
Musisz na graczach wywżeć presję by z nim poszli.
Jeśli nie będą chcieli z nim pójść daj im pobłądzić jeszcze chwilę i znaleźć jego chate.
\marginnote{|Miejsce: Chata wolnego Jelenia}
Jest to duża, jednoizbowa drewniana chata w stylu ``traperskim''.
Miejsce jest bardzo przytulne, zapach gulaszu z dziczyzny i ciepło kominka poprawia nastruj.
Jeleń wchodząc wita serdecznie żonę i dwoje dzieciaków.
Prosi ją po indiańsku o coś, a postaciom graczy poleca zasiąść za dość sporym stołem.
Po chwili Pan domu kładzie na stole dzban herbaty, to jest ta chwila gdy robisz to samo jako mistrz gry.
Ten mały trik ma spowodować by faktycznie poczuli się pewnie.
Gdy gracze rozmarzają przy herbatce Jeleń przeprasza ich, znika na zewnąrz i po chwili wraca ze zdobyczami.
Siada w kącie chaty i zaczyna rozmowe z postaciami graczy, skórując króliki.
Na stół trafia gulasz, po dobrej gorącej porcji dla każdego.
Jeleń płynnie przechodzi do historii o Pakaczuki.


%\label{hand:testament}
\end{document}
